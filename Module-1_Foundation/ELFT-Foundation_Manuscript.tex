\documentclass[12pt, a4paper]{article}
\usepackage[utf8]{inputenc}
\usepackage{amsmath, amssymb, amsthm}
\usepackage{physics}
\usepackage{graphicx}
\usepackage{hyperref}

\title{Foundations of the Electromagnetic-like Field Theory (ELFT)}
\author{Your Name \\ and the ELFT Collaboration}
\date{\today}

\begin{document}

\maketitle

\begin{abstract}
    This manuscript presents the foundational framework of the Electromagnetic-like Field Theory (ELFT). By introducing a scalar field $\psi$ and a vector field $\mathbf{B}$ as fundamental descriptors of spacetime, we construct a Lagrangian density from which field equations analogous to those of Maxwell and Klein-Gordon are naturally derived. This framework suggests a unified geometric origin for properties traditionally considered separate, such as mass and spacetime curvature, potentially offering new pathways toward the unification of physical interactions.
\end{abstract}

\section{Introduction}
\subsection{Motivation: The Problem of Constants}
The origin of the fundamental constants of physics (e.g., $c$, $G$, $\varepsilon_0$) remains one of the most profound open questions. Are they truly fixed, or could they be emergent properties of a deeper dynamical structure? The ELFT is conceived from the proposition that the latter may be true.

\section{The Fundamental Fields and Lagrangian}

\subsection{Definition of Core Fields}
The theory is built upon two primary fields defined on spacetime:
\begin{itemize}
    \item A scalar field $\psi(\mathbf{r}, t)$
    \item A vector field $\mathbf{B}(\mathbf{r}, t)$, termed the \textit{proto-magnetic} field
\end{itemize}

\subsection{The Lagrangian Formulation}
The dynamics of the system are governed by the following Lagrangian density:
\[
\mathcal{L} = T - V(\mathbf{r}) - E(\mathbf{r},t) = \frac{1}{2}m_0 \dot{\mathbf{r}}^2 - \psi(\mathbf{r},t)^2 - V(\mathbf{r})
\]
where $V(\mathbf{r}, t) = \mathbf{B}(\mathbf{r},t)^2$ is identified as an effective potential.

\subsection{From Particle Action to Field Equations}
Applying the Euler-Lagrange equations to the particle Lagrangian:
\[
\frac{d}{dt}\left( \frac{\partial L}{\partial \dot{r}_i} \right) - \frac{\partial L}{\partial r_i} = 0
\]
yields the equation of motion:
\begin{equation}
m_0 \ddot{\mathbf{r}} + 2\psi \nabla \psi + \nabla V = 0.
\end{equation}

\section{Analogy with Established Field Theories}

\subsection{The Klein-Gordon Connection}
By transitioning from a particle description to a field description, promoting $\psi$ to a relativistic field, we can derive a Klein-Gordon-type equation:
\begin{equation}
\left( \Box - k^2 \right)\psi = 0,
\end{equation}
where $\Box$ is the d'Alembertian operator.

\subsection{Tensor Formulation and Maxwell Analogy}
We define an antisymmetric tensor $F^{\mu\nu}$ analogous to the electromagnetic field tensor:
\[
F^{\mu\nu} = \begin{pmatrix}
0 & -\psi_x/c & -\psi_y/c & -\psi_z/c \\
\psi_x/c & 0 & -B_z & B_y \\
\psi_y/c & B_z & 0 & -B_x \\
\psi_z/c & -B_y & B_x & 0
\end{pmatrix}.
\]
This allows us to write the field equations in a compact, covariant form reminiscent of Maxwell's equations.

\section{Discussion and Outlook}
The ELFT framework presented here demonstrates that a theory built upon the fields $(\psi, \mathbf{B})$ can naturally reproduce structures central to modern physics. The next crucial steps involve...
% ... (We will expand this with the connections to GR and the Higgs mechanism in the next iteration)

\section*{Acknowledgments}
This work is dedicated to all who seek light in the darkness and persevere with hope. The journey of this theory is intertwined with the resilience of the human spirit.

\end{document}
